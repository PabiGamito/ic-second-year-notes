\documentclass[twocolumn,english]{article}
\renewcommand{\familydefault}{\sfdefault}
\usepackage[latin9]{inputenc}
\usepackage[landscape]{geometry}
\geometry{verbose,tmargin=0.5in,bmargin=0.75in,lmargin=0.5in,rmargin=0.5in}
\setlength{\parskip}{0bp}
\setlength{\parindent}{0pt}
\usepackage{color}
\usepackage{float}
\usepackage{amsmath}
\usepackage{amssymb}
\usepackage{esint}

\makeatletter

\providecommand{\tabularnewline}{\\}








\providecommand{\tabularnewline}{\\}




\usepackage{array}
\usepackage{multirow}





\providecommand{\tabularnewline}{\\}

\setlength{\columnsep}{0.25in}
\usepackage{xcolor}
\usepackage{textcomp}
\usepackage{listings}
\lstset{
  tabsize=2,
  basicstyle=\small\ttfamily,
}



\usepackage{babel}
\usepackage{listings}
\renewcommand{\lstlistingname}{Listing}


\usepackage{titlesec}
\titleformat*{\section}{\color{blue!60!green!40!black} \vspace{8pt}\titlerule\vspace{4pt}\Large\bfseries\sffamily}
\titleformat*{\subsection}{\color{blue!60!green} \vspace{2pt}\large\bfseries\sffamily}



\usepackage{enumitem}
\setlist{itemsep=0pt}


\let\emph\relax
\DeclareTextFontCommand{\emph}{\bfseries}

\definecolor{red}{rgb}{0.5,0.5,0.5}

\usepackage{babel}

\makeatother

\usepackage{babel}
\begin{document}

\title{\vspace{-4ex}
Revision Notes for CO245 Probability and Statistics\vspace{-4ex}
}


\date{Spring 2018\vspace{-2ex}
}

\maketitle

\paragraph{Results in \textcolor{red}{grey} are included in the formula sheet.}


\section{Probability}


\paragraph{Sample Spaces and Events}
\begin{itemize}
\item \emph{Sample space} $S$: Range of possible outcomes of a random experiment. 
\item \emph{Event}: Subset of sample space. 

\begin{itemize}
\item \emph{Null event}: $\emptyset$. 
\end{itemize}
\item Events are \emph{mutually exclusive} if $\forall i,j.E_{i}\cap E_{j}=\emptyset$. 
\end{itemize}

\paragraph{The $\sigma$-algebra}

A collection $\mathfrak{S}$ of subsets of $S$ is a $\sigma$-field
or $\sigma$-algebra if it satisfies: 
\begin{enumerate}
\item Nonempty: $S\in\mathfrak{S}$. 
\item Closed under complements: if $E\in\mathfrak{S}$ then $\overline{E}\in\mathfrak{S}$. 
\item Closed under countable union: if $E_{1},E_{2},\dots\in\mathfrak{S}$
then $\cup_{i=1}^{\infty}E_{i}\in\mathfrak{S}$. 
\end{enumerate}
\emph{Axioms}: 
\begin{enumerate}
\item For any $E$ in $\mathfrak{S}$, $0\leq P\left(E\right)\leq1$. 
\item $P\left(S\right)=1$. 
\item Countably additive: $P\left(\cup_{i}E_{i}\right)=\sum_{i}P\left(E_{i}\right)$. 
\end{enumerate}
\emph{Properties}: 
\begin{enumerate}
\item $P\left(\overline{E}\right)=1-P\left(E\right)$. 
\item $P\left(\emptyset\right)=0$. 
\item \textcolor{red}{For any events $E$ and $F$, $P\left(E\cup F\right)=P\left(E\right)+P\left(F\right)-P\left(E\cap F\right)$. }
\end{enumerate}

\paragraph{Independence}
\begin{itemize}
\item \textcolor{red}{Events $E$ and $F$ are independent iff $P\left(E\cap F\right)=P\left(E\right)P\left(F\right)$. }
\item If $E$ and $F$ are independent, $\overline{E}$ and $F$ are also
independent. 
\end{itemize}

\paragraph{Conditional Probability}
\begin{enumerate}
\item $P\left(E\mid F\right)$ is called a \emph{conditional} probability. 
\item $P\left(E\cap F\right)$ is called a \emph{joint} probability. 
\item $P\left(E\right)$ is called a \emph{marginal} probability. 
\end{enumerate}
\textrm{\textcolor{red}{
\[
P\left(E\mid F\right)=\frac{P\left(E\cap F\right)}{P\left(F\right)}
\]
}}
\begin{itemize}
\item Events $E_{1}$ and $E_{2}$ are \emph{conditionally independent}
given $F$ iff $P\left(E_{1}\cap E_{2}\mid F\right)=P\left(E_{1}\mid F\right)P\left(E_{2}\mid F\right)$. 
\item \emph{Bayes theorem} (easily derived from definition above) states:
\[
P\left(E\mid F\right)=\frac{P\left(E\right)P(F\mid E)}{P\left(F\right)}
\]
\emph{Read the question} and \emph{check carefully}. Make sure you
know the difference between $P\left(A\cap B\right)$ and $P\left(A\mid B\right)$!
\item For a set of events $\left\{ F_{1},F_{2},\dots\right\} $ which form
a partition of $S$, the \emph{partition rule} (derived from $E=E\cap S$)
states: 
\[
P\left(E\right)=\sum_{i}P\left(E\mid F_{i}\right)P\left(F_{i}\right)
\]

\end{itemize}

\paragraph{Likelihood and Posterior Probability}

For parameters $\theta$ and evidence $X$: 
\begin{enumerate}
\item \emph{Likelihood function} is $P\left(X\mid\theta\right)$. 
\item \emph{Posterior probability} is $P\left(\theta\mid X\right)$. 
\item \emph{Prior probability} is $P\left(\theta\right)$. 
\end{enumerate}
By Bayes theorem:

\[
P\left(\theta\mid X\right)=\frac{P\left(X\mid\theta\right)P\left(\theta\right)}{P\left(X\right)}
\]



\section{Random Variables}

Mapping from sample space to $\mathbb{R}$ (e.g. $X:S\rightarrow\mathbb{R}$). 
\begin{itemize}
\item \emph{Probability distribution function} $P_{X}\left(x\right)=P\left(X^{-1}\left(x\right)\right)$. 

\begin{itemize}
\item Probabilities are between 0 and 1. 
\item Sum to 1. 
\end{itemize}
\item \emph{Cumulative distribution function} $F_{X}\left(x\right)=P_{X}\left(X\leq x\right)$. 

\begin{itemize}
\item For every real number $x$, $0\leq F_{X}\left(x\right)\leq1$. 
\item $F_{X}$ is monotonic. 
\item $F_{X}\left(-\infty\right)=0$ and $F_{X}\left(\infty\right)=1$. 
\end{itemize}
\item A random variable is \emph{simple} iff it can only take a finite number
of possible values. 
\end{itemize}

\subsection{Discrete Random Variables}


\subsubsection{Definition}

$X$ is discrete iff $\text{range}\left(X\right)$ is countable. 
\[
p\left(x_{i}\right)=F\left(x_{i}\right)-F\left(x_{i-1}\right)
\]
\[
F\left(x_{i}\right)=\sum_{j=1}^{i}p\left(x_{j}\right)
\]

\begin{itemize}
\item $p_{X}$ is the \emph{probability mass function}. 
\item $F_{x}$ is the \emph{cumulative distribution function}. 
\end{itemize}

\subsubsection{Expectation and Probability Generating Function}


\paragraph{Mean $E\left(X\right)$}

\textrm{\textcolor{red}{
\[
E_{X}\left(X\right)=\sum_{x}xp_{X}\left(x\right)
\]
}}

\textrm{\textcolor{red}{
\[
E_{X}\left(g\left(X\right)\right)=\sum_{x}g\left(x\right)p_{X}\left(x\right)
\]
}}

\[
E_{X}\left(aX+b\right)=aE_{X}\left(X\right)+b
\]



\paragraph{Variance $\text{Var}\left(X\right)$}

\textrm{\textcolor{red}{
\[
\text{Var}_{X}\left(X\right)=E_{X}\left[\left(X-E_{X}\left(X\right)\right)^{2}\right]=E\left(X^{2}\right)-\left(E\left(X\right)\right)^{2}
\]
}}

\[
\text{Var}\left(aX+b\right)=a^{2}\text{Var}\left(X\right)
\]



\paragraph{Standard Deviation $\text{sd}_{X}\left(X\right)$}

\textrm{\textcolor{red}{
\[
\text{sd}_{X}\left(X\right)=\sqrt{\text{Var}_{X}\left(X\right)}
\]
}}

\emph{Don't forget the square root!}


\paragraph{Skewness $\gamma_{1}$}

\textrm{\textcolor{red}{
\[
\gamma_{1}=\frac{E_{X}\left[\left(X-E_{X}\left(X\right)\right)^{3}\right]}{\text{sd}_{X}\left(X\right)^{3}}
\]
}}


\paragraph{Probability Generating Function $G_{X}\left(z\right)$}

\[
G_{X}\left(z\right)=E_{X}\left(z^{X}\right)=\sum_{x}p_{X}\left(x\right)z^{x}
\]


Look out for well-known series results (e.g. Maclaurin series / geometric
series).


\paragraph{Moments $M_{n}$}
\begin{itemize}
\item The $n$th \emph{moment} of a random variable $X$ is $M_{n}=E\left(X^{n}\right)$. 
\item The $n$th \emph{factorial moment} is $M_{n}^{f}=E\left(X\left(X-1\right)...\left(X-n+1\right)\right)=G^{(n)}\left(1\right)$.
\begin{align*}
M_{0} & = & M_{0}^{f} & = & G\left(1\right)=1\\
M_{1} & = & M_{1}^{f} & = & G'\left(1\right)\\
M_{2} & = & M_{2}^{f}+M_{1}^{f} & = & G''\left(1\right)+G'\left(1\right)
\end{align*}

\end{itemize}

\paragraph{Sums of Random Variables}

Where $S_{n}=\sum_{i=1}^{n}X_{i}$ is a sum of random variables, and
$S_{n}/n$ is their average:

\begin{align*}
E\left(S_{n}\right)=\sum_{i=1}^{n}E\left(X_{i}\right) &  & E\left(\frac{S_{n}}{n}\right)=\frac{\sum_{i=1}^{n}E\left(X_{i}\right)}{n}\\
\text{Var}\left(S_{n}\right)=\sum_{i=1}^{n}\text{Var}\left(X_{i}\right) &  & \text{Var}\left(\frac{S_{n}}{n}\right)=\frac{\sum_{i=1}^{n}\text{Var}\left(X_{i}\right)}{n^{2}} &  & \left(X_{i}\text{ are independent}\right)\\
G_{S_{n}}\left(z\right)=\prod_{i=1}^{n}G_{X_{i}}\left(z\right) &  &  &  & \left(X_{i}\text{ are independent}\right)
\end{align*}



\subsubsection{Discrete Distributions}


\paragraph{Bernoulli $\text{Bernoulli}\left(p\right)$}
\begin{itemize}
\item $p\left(x\right)=p^{x}\left(1-p\right)^{1-x}$ for $x=0,1$. 
\item $\mu=p$. 
\item $\sigma^{2}=p\left(1-p\right)$. 
\end{itemize}

\paragraph{Binomial $\text{Binomial}\left(n,p\right)$}
\begin{itemize}
\item $n$ identical \emph{independent} $\text{Bernoulli}\left(p\right)$
trials. 
\item \textrm{\textcolor{red}{$p\left(x\right)=\left(\begin{array}{c}
n\\
x
\end{array}\right)p^{x}\left(1-p\right)^{n-x}$ for $x=1,2,\dots,n$.}} (Remember $\left(\begin{array}{c}
n\\
x
\end{array}\right)=\frac{n!}{x!\left(n-x\right)!}$).
\item \textrm{\textcolor{red}{$\mu=np$. }}
\item \textrm{\textcolor{red}{$\sigma^{2}=np\left(1-p\right)$.}}\textrm{\textcolor{green}{{} }}
\item $\gamma_{1}=\frac{1-2p}{\sqrt{np\left(1-p\right)}}$. 
\end{itemize}

\paragraph{Geometric $\text{Geometric}\left(p\right)$}
\begin{itemize}
\item Potentially infinite sequence of \emph{independent} $\text{Bernoulli}\left(p\right)$
trials. 
\item \textrm{\textcolor{red}{$p\left(x\right)=p\left(1-p\right)^{x-1}$
for $x=1,2,\dots$. }}
\item \textrm{\textcolor{red}{$\mu=\frac{1}{p}$. }}
\item \textrm{\textcolor{red}{$\sigma^{2}=\frac{1-p}{p^{2}}$. }}
\item $\gamma_{1}=\frac{2-p}{\sqrt{1-p}}$. 
\end{itemize}

\paragraph{Poisson $\text{Poi}\left(\lambda\right)$}
\begin{itemize}
\item \textrm{\textcolor{red}{$p\left(x\right)=\frac{e^{-\lambda}\lambda^{x}}{x!}$
for $x=0,1,2,\dots$. }}
\item \textrm{\textcolor{red}{$\mu=\sigma^{2}=\lambda$. }}
\item $\gamma_{1}=\frac{1}{\sqrt{\lambda}}$. 
\item $G\left(z\right)=e^{-\lambda\left(1-z\right)}$. 
\item When $p$ is small and $n$ is large, $\text{Binomial}\left(n,p\right)$
is approximated by $\text{Poi}\left(n,p\right)$. 
\end{itemize}

\paragraph{Uniform $\text{U}\left(\left\{ 1,2,\dots,n\right\} \right)$}
\begin{itemize}
\item \textrm{\textcolor{red}{$p\left(x\right)=\frac{1}{n}$ for $x=1,2,\dots,n$. }}
\item \textrm{\textcolor{red}{$\mu=\frac{n+1}{2}$. }}
\item \textrm{\textcolor{red}{$\sigma^{2}=\frac{n^{2}-1}{12}$. }}
\item $\gamma_{1}=0$. 
\end{itemize}

\subsection{Continuous Random Variables}


\subsubsection{Definition}

$X$ is a \emph{continuous random variable} if $\exists f_{X}:\mathbb{R}\rightarrow\mathbb{R}$
s.t. 
\[
P_{X}\left(B\right)=\int_{x\in B}f_{X}\left(x\right)\text{d}x
\]

\begin{itemize}
\item $f_{X}$ is the \emph{probability density function}. 
\item The \emph{cumulative distribution function} is \textrm{\textcolor{red}{
\[
F_{X}\left(x\right)=P\left(X\leq x\right)=\int_{-\infty}^{x}f_{X}\left(t\right)\text{d}t
\]
}}
\item Note that \textcolor{red}{$f_{X}\left(x\right)=F_{X}^{\prime}\left(x\right)$}. 
\end{itemize}

\paragraph{Properties of a pdf}
\begin{enumerate}
\item For all $x\in\mathbb{R},$ $f_{X}\left(x\right)\geq0$. 
\item $\int_{-\infty}^{\infty}f_{X}\left(x\right)\text{d}x=1$. 
\end{enumerate}

\paragraph{Transformed Random Variables}

E.g. $Y=g\left(X\right)$ for some $g:\mathbb{R}\rightarrow\mathbb{R}$
where $g$ is \emph{continuous} and \emph{strictly monotonic} (so
it has an inverse). 
\begin{itemize}
\item $F_{Y}\left(y\right)=P\left(Y\leq y\right)=P\left(g\left(X\right)\leq Y\right)=P\left(X\leq g^{-1}\left(Y\right)\right)=F_{X}\left(g^{-1}\left(y\right)\right)$.
\item By the chain rule, we get $f_{Y}\left(y\right)=F_{Y}^{\prime}\left(y\right)=f_{X}\left(g^{-1}\left(y\right)\right)\left|g^{-1'}\left(y\right)\right|$. 
\end{itemize}

\subsubsection{Mean, Variance and Quantiles}


\paragraph{Mean $E\left(X\right)$}

\textrm{\textcolor{red}{
\[
E_{X}\left(X\right)=\int_{-\infty}^{\infty}xf_{X}\left(x\right)\text{d}x
\]
}}

\[
E_{X}\left(g\left(X\right)\right)=\int_{-\infty}^{\infty}g\left(x\right)f_{X}\left(x\right)\text{d}x
\]


Properties: 
\begin{enumerate}
\item \emph{Linearity}: $E\left(aX+b\right)=aE\left(X\right)+b$. 
\item \emph{Additivity}: $E\left(g\left(X\right)+h\left(X\right)\right)=E\left(g\left(X\right)\right)+E\left(h\left(X\right)\right)$. 
\end{enumerate}

\paragraph{Variance $\text{Var}\left(X\right)$}

\begin{multline*}
{\color{red}\text{Var}_{X}\left(X\right)}=E\left(\left(X-\mu_{X}\right)^{2}\right)\\
=\int_{-\infty}^{\infty}\left(x-\mu_{X}\right)^{2}f_{X}\left(x\right)\text{d}x\\
=\int_{-\infty}^{\infty}x^{2}f_{X}\left(x\right)\text{d}x-\mu_{X}^{2}\\
{\color{red}=E\left(X^{2}\right)-\left(E\left(X\right)\right)^{2}}
\end{multline*}

\begin{itemize}
\item $\text{Var}\left(aX+b\right)=a^{2}\text{Var}\left(X\right)$. 
\end{itemize}

\paragraph{Moment Generating Function $M_{X}\left(t\right)$}

\[
M_{X}\left(t\right)=E\left(e^{tX}\right)=\int_{-\infty}^{\infty}e^{tx}f_{X}\left(x\right)\text{d}x
\]

\begin{itemize}
\item Might not exist (for some $t$). 
\item The $n$th moment is $\left.M_{n}=\frac{\text{d}^{n}M_{X}\left(t\right)}{\text{d}t^{n}}\right\vert _{t=0}$. 
\end{itemize}

\paragraph{Characteristic Function $\phi_{X}\left(t\right)$}

\[
\phi_{X}\left(t\right)=E\left(e^{itX}\right)=\int_{-\infty}^{\infty}e^{itx}f_{X}\left(x\right)\text{d}x
\]

\begin{itemize}
\item Always exists (Fourier transform of pdf). 
\item The $n$th moment is $\left.M_{n}=\left(-i\right)^{n}\frac{\text{d}^{n}\phi_{X}\left(t\right)}{\text{d}t^{n}}\right\vert _{t=0}$. 
\end{itemize}

\paragraph{Probability Generating Functions}

\[
M\left(t\right)=G\left(e^{t}\right)\text{ and }\phi\left(t\right)=G\left(e^{it}\right)
\]



\paragraph{Sum of Random Variables}

For independent random variables $X_{1},X_{2},\dots,X_{n}$, and $S_{n}=\sum_{j=1}^{n}X_{j}$:
\[
\phi_{S_{n}}\left(t\right)=\prod_{j=1}^{n}\phi_{X_{j}}\left(t\right)\text{ and }M_{S_{n}}\left(t\right)=\prod_{j=1}^{n}M_{X_{j}}\left(t\right)
\]



\paragraph{Quantiles}

\[
Q_{X}\left(\alpha\right)=F_{X}^{-1}\left(\alpha\right)
\]


E.g. \emph{median} is $F_{X}^{-1}\left(\frac{1}{2}\right)$. I.e.
the solution to $F_{X}\left(x\right)=\frac{1}{2}$.


\subsubsection{Continuous Distributions}


\paragraph{Uniform $\text{U}\left(a,b\right)$}
\begin{itemize}
\item \textrm{\textcolor{red}{$f\left(x\right)=\begin{cases}
\frac{1}{b-a} & a<x<b\\
0 & \text{otherwise}
\end{cases}$. }}
\item $F\left(x\right)=\begin{cases}
0 & x\leq a\\
\frac{x-a}{b-a} & a<x<b\\
0 & x\geq b
\end{cases}$. 
\item \textrm{\textcolor{red}{$\mu=\frac{a+b}{2}$. }}
\item \textrm{\textcolor{red}{$\sigma^{2}=\frac{\left(b-a\right)^{2}}{12}$. }}
\end{itemize}

\paragraph{Exponential $\text{Exp}\left(\lambda\right)$}
\begin{itemize}
\item \textrm{\textcolor{red}{$f\left(x\right)=\lambda e^{-\lambda x}$
for $x\geq0$. }}
\item $F\left(x\right)=1-e^{-\lambda x}$ for $x\geq0$. 
\item \textrm{\textcolor{red}{$\mu=\frac{1}{\lambda}$. }}
\item \textrm{\textcolor{red}{$\sigma^{2}=\frac{1}{\lambda^{2}}$. }}
\item \emph{Memoryless}: $P\left(X>x\right)=e^{-\lambda x}$ and $P\left(X>x+s\mid X>s\right)=e^{-\lambda x}$. 
\item If the number of events is distributed by $N\sim\text{Poi}\left(\lambda\right)$
then the time between consecutive events is distributed by $T\sim\text{Exp}\left(\lambda\right)$. 
\end{itemize}

\paragraph{Normal $\text{N}\left(\mu,\sigma^{2}\right)$}
\begin{itemize}
\item \textrm{\textcolor{red}{$f\left(x\right)=\frac{1}{\sigma\sqrt{2\pi}}e^{-\frac{\left(x-\mu\right)^{2}}{2\sigma^{2}}}$. }}
\item $F\left(x\right)=\frac{1}{\sigma\sqrt{2\pi}}\int_{-\infty}^{x}e^{-\frac{\left(t-\mu\right)^{2}}{2\sigma^{2}}}dt$. 
\end{itemize}
\emph{Standard Normal}: 
\begin{itemize}
\item $Z\sim\text{N}\left(0,1\right)$. 
\item $f\left(z\right)$ is written as $\phi\left(z\right)$ and $F\left(z\right)$
as $\Phi\left(z\right)$. 
\item $\phi\left(-z\right)=\phi\left(z\right)$ and $\Phi\left(z\right)=1-\Phi\left(-z\right)$. 
\end{itemize}
\emph{Standardising Normal RVs}: 
\begin{itemize}
\item \textrm{\textcolor{red}{$X\sim\text{N}\left(\mu,\sigma^{2}\right)\implies\frac{X-\mu}{\sigma}\sim\text{N}\left(0,1\right)$. }}
\end{itemize}
\emph{Central Limit Theorem}: 
\begin{itemize}
\item For $\overline{X}=\frac{\sum_{i=1}^{n}X_{i}}{n}$ where $X_{1},X_{2},\dots,X_{n}$
are independent and identically distributed random variables, $\lim_{n\rightarrow\infty}\frac{\overline{X}-\mu}{\sigma/\sqrt{n}}\sim\text{N}\left(0,1\right)$. 
\item I.e. for large $n$, $\overline{X}\sim N\left(\mu,\frac{\sigma^{2}}{n}\right)$. 
\item For large $n$, $\text{Binomial}\left(n,p\right)\approx\text{N}\left(np,np\left(1-p\right)\right)$ 
\end{itemize}
\emph{Log-Normal Distribution}: 
\begin{itemize}
\item $Y$ is said to follow a log-normal distribution if $Y=e^{X}$ and
$X\sim\text{N}\left(\mu,\sigma^{2}\right)$. 
\end{itemize}

\subsection{Joint Random Variables}


\paragraph{Definitions}

Has joint cdf: 
\[
F_{XY}\left(x,y\right)=P_{XY}\left(\left(-\infty,x\right],\left(-\infty,y\right]\right)
\]


We can recover marginal cdfs: 
\[
F_{X}\left(x\right)=F_{XY}\left(x,\infty\right)
\]
\[
F_{Y}\left(y\right)=F_{XY}\left(\infty,y\right)
\]


Has joint pmf:

\[
p_{XY}\left(x,y\right)=P_{XY}\left(X=x,Y=y\right)
\]
We can recover marginal pmfs:

\textrm{\textcolor{red}{
\[
p_{X}\left(x\right)=\sum_{y}p_{XY}\left(x,y\right)
\]
}}

\textrm{\textcolor{red}{
\[
p_{Y}\left(y\right)=\sum_{x}p_{XY}\left(x,y\right)
\]
}}


\paragraph{Definitions for Jointly Continuous Variables}

Has joint cdf:

\textrm{\textcolor{red}{
\[
F_{XY}\left(x,y\right)=\int_{t=-\infty}^{y}\int_{s=-\infty}^{x}f_{XY}\left(s,t\right)\text{d}s\text{d}t
\]
}}

Has joint pdf:

\textrm{\textcolor{red}{
\[
f_{XY}\left(x,y\right)=\frac{\partial^{2}}{\partial x\partial y}F_{XY}\left(x,y\right)
\]
}}We can recover marginal pdfs:

\textrm{\textcolor{red}{
\[
f_{X}\left(x\right)=\int_{y=-\infty}^{\infty}f_{XY}\left(x,y\right)\text{d}y
\]
}}

\textrm{\textcolor{red}{
\[
f_{Y}\left(y\right)=\int_{x=-\infty}^{\infty}f_{XY}\left(x,y\right)\text{d}x
\]
}}

\emph{Take care with integration}. E.g. don't ignore constants!


\paragraph{Conditional Distributions}

\textrm{\textcolor{red}{
\[
p_{Y\mid X}\left(y\mid x\right)=\frac{p_{XY}\left(x,y\right)}{p_{X}\left(x\right)}
\]
}}


\paragraph{Finding Probability of $X<Y$}

\begin{multline*}
P\left(X<Y\right)=\int_{y=-\infty}^{\infty}F_{X|Y}\left(y\mid y\right)f_{Y}\left(y\right)\mbox{d}y\\
=\int_{y=-\infty}^{\infty}\int_{x=-\infty}^{y}f_{XY}\left(x,y\right)\text{d}x\text{d}y
\end{multline*}



\paragraph{Expectation}

\[
E_{XY}\left(g\left(x,y\right)\right)=\sum_{y}\sum_{x}g\left(x,y\right)p_{XY}\left(x,y\right)
\]


\[
E_{XY}\left(g\left(x,y\right)\right)=\int_{y=-\infty}^{\infty}\int_{x=-\infty}^{\infty}g\left(x,y\right)p_{XY}\left(x,y\right)\text{d}x\text{d}y
\]



\paragraph{Conditional Expectation}

\[
E_{Y\mid X}\left(Y|X=x\right)=\sum_{y}y\:p_{Y|X}\left(y|x\right)
\]


\[
E_{Y\mid X}\left(Y|X=x\right)=\int_{y=-\infty}^{\infty}y\:f_{Y|X}\left(y|x\right)\text{d}y
\]



\paragraph{Tower Rule}

\[
E_{Y}\left(Y\right)=E_{X}\left(E_{Y\mid X}\left(Y\mid X\right)\right)
\]



\paragraph{Covariance}

Measures how two RVs change in tandem with one another.

\textrm{\textcolor{red}{
\[
\sigma_{XY}=E_{XY}\left(\left(X-\mu_{X}\right)\left(Y-\mu_{Y}\right)\right)
\]
}}


\paragraph{Correlation}

Invariant to scale of the RVs $X$ and $Y$.

\textrm{\textcolor{red}{
\[
\rho_{XY}=\frac{\sigma_{XY}}{\sigma_{X}\sigma_{Y}}
\]
}}


\section{Estimation}

For $\underline{X}=\left(X_{1},\dots,X_{n}\right)$ representing $n$
iid data samples from a population with distribution $P_{X}$, we
observe $\underline{x}=\left(x_{1},\dots,x_{n}\right)$.


\paragraph{Statistic}

A random variable:

\[
T=T\left(X_{1},\dots,X_{n}\right)=T\left(\underline{X}\right)
\]


Observed statistic is $t=t\left(\underline{x}\right)$.


\paragraph{Estimator}

A statistic $T\left(\underline{X}\right)$ when used to approximate
parameters of the distribution $P_{X\mid\theta}\left(x\mid\theta\right)$.
An \emph{estimate} is the realised value for the estimator for a particular
sample $t\left(\underline{x}\right)$.


\paragraph{Bias}

Of an estimator $T$ for a parameter $\theta$:

\textrm{\textcolor{red}{
\[
\text{bias}\left(T\right)=E\left(T\mid\theta\right)-\theta
\]
}}
\begin{itemize}
\item \emph{Unbiased} if bias is 0. 
\item Sample mean $\overline{x}$ is an unbiased estimate for population
mean $\mu$. 
\item Biased-corrected sample variance $S_{n-1}^{2}=\frac{n}{n-1}S^{2}$
is an unbiased estimate for $\sigma^{2}$. 
\end{itemize}

\paragraph{Efficiency}

For two unbiased estimators, $\hat{T}$ and $\tilde{T}$, $\hat{T}$
is \emph{more efficient} than $\tilde{T}$ if: 
\begin{enumerate}
\item For all $\theta$, $\text{Var}_{\hat{T}\mid\theta}\left(\hat{T}\mid\theta\right)\leq\text{Var}_{\tilde{T}\mid\theta}\left(\tilde{T}\mid\theta\right)$,
and 
\item There is some $\theta$ with $\text{Var}_{\hat{T}\mid\theta}\left(\hat{T}\mid\theta\right)<\text{Var}_{\tilde{T}\mid\theta}\left(\tilde{T}\mid\theta\right)$. 
\end{enumerate}
$\hat{T}$ is \emph{efficient} if it is more efficient that any other
possible estimator.


\paragraph{Consistency}

An estimator $T$ is \emph{consistent} if 
\[
\forall\epsilon>0\qquad P\left(\left|T-\theta\right|>\epsilon\right)\rightarrow0\text{ as }n\rightarrow\infty
\]


If $T$ is unbiased and $\lim_{n\rightarrow\infty}\text{Var}\left(T\right)=0$
then $T$ is consistent.


\subsection{Maximum Likelihood Estimation}
\begin{enumerate}
\item Find the likelihood function $L\left(\theta\right)$ where: \textrm{\textcolor{red}{
\[
L\left(\theta\mid\underline{x}\right)=\prod_{i=1}^{n}p_{X\mid\theta}\left(x_{i}\right)\text{ or }\prod_{i=1}^{n}f_{X\mid\theta}\left(x_{i}\right)
\]
}}
\item Take the natural log of the likelihood $l\left(\theta\mid\underline{x}\right)$,
and collect terms involving $\theta$: 
\[
l\left(\theta\mid\underline{x}\right)=\sum_{i=1}^{n}\log\left(p_{X\mid\theta}\left(x_{i}\right)\right)\text{ or }\sum_{i=1}^{n}\log\left(f_{X\mid\theta}\left(x_{i}\right)\right)
\]

\item Find the value of $\theta$ for which log-likelihood is maximised:
usually find the $\hat{\theta}$ that solves 
\[
\frac{\delta}{\delta\theta}l\left(\hat{\theta}\right)=\frac{\delta}{\delta\theta}\log\left(L\left(\hat{\theta}\right)\right)=0
\]

\item Ensure that the estimate $\hat{\theta}$ corresponds to a maximum
by checking that the second derivative satisfies 
\[
\frac{\partial^{2}}{\partial\theta^{2}}l\left(\hat{\theta}\right)<0
\]

\end{enumerate}
The MLE is not necessarily unbiased, but it is consistent and efficient,
if an efficient estimator exists.


\paragraph{Confidence Intervals}

The $100\left(1-\alpha\right)\%$ confidence interval for $\mu$ is
given by: \textrm{\textcolor{red}{
\[
\left[\overline{x}-z_{1-\frac{\alpha}{2}}\frac{\sigma}{\sqrt{n}},\overline{x}+z_{1-\frac{\alpha}{2}}\frac{\sigma}{\sqrt{n}}\right]
\]
}}
\begin{itemize}
\item \emph{Normal distribution with known variance}: confidence interval
above is exact. 
\item \emph{Other distributions}: confidence interval above is an approximate,
by CLT. 
\item \emph{Normal distribution with unknown variance}: need to use bias-corrected
variance: exact CI is given by ${\color{red}\left[\overline{x}-t_{n-1,1-\frac{\alpha}{2}}\frac{S}{\sqrt{n-1}},\overline{x}+t_{n-1,1-\frac{\alpha}{2}}\frac{S}{\sqrt{n-1}}\right]}$. 
\end{itemize}

\subsection{Bayesian Estimation}


\paragraph{Maximum Likelihood Estimator}
\begin{itemize}
\item Doesn't take into account any prior information about the MLE. 
\item Only returns a single and specific value of $\theta$. 
\end{itemize}

\paragraph{Using Prior Information}

Use Bayes theorem 
\[
\underbrace{P\left(\theta\mid X\right)}_{\text{posterior}}=\underbrace{P\left(X\mid\theta\right)}_{\text{likelihood}}\times\underbrace{P\left(\theta\right)}_{\text{prior}}\times\underbrace{\frac{1}{P\left(X\right)}}_{\text{evidence}}
\]



\paragraph{Maximum a Posteriori Estimator}

Instead of maximising $\prod_{i=1}^{n}P\left(X=x_{i}|\theta\right)$,
maximise $\prod_{i=1}^{n}P\left(\theta\mid X=x_{i}\right)=\prod_{i=1}^{n}P\left(X=x_{i}\mid\theta\right)\times P\left(\theta\right)$.


\paragraph{Prior Distributions}

Often we use the \emph{beta distribution}: $\text{Beta}\left(\theta;\alpha,\beta\right)=\frac{1}{B\left(\alpha,\beta\right)}\times\theta^{\alpha-1}\left(1-\theta\right)^{\beta-1}$
with $B\left(\alpha,\beta\right)=\frac{\Gamma\left(\alpha\right)\Gamma\left(\beta\right)}{\Gamma\left(\alpha+\beta\right)}$
and $\Gamma\left(z\right)=\int_{0}^{\infty}x^{z-1}e^{-x}\text{d}x$


\section{Hypothesis Testing}


\subsection{Testing Population Mean}


\paragraph{Hypotheses}
\begin{itemize}
\item $H_{0}:\theta=\theta_{0}$ vs. $H_{1}:\theta\neq\theta_{0}$ is a
\emph{two-sided} test. 
\item $H_{0}:\theta>\theta_{0}$ vs. $H_{1}:\theta<\theta_{0}$ is a \emph{one-sided}
test. 
\end{itemize}

\paragraph{Tests Statistics and Rejection Regions}
\begin{itemize}
\item Choose a test statistic $T\left(\underline{X}\right)$ for which we
can find the distribution under $H_{0}$. 
\item Calculate the rejection region $R$ such that $P\left(T\in R\mid H_{0}\right)=\alpha$
for some small probability $\alpha$. 
\item Compute the realised value of the test statistic and conclude appropriately. 
\end{itemize}

\paragraph{Errors and Power}
\begin{enumerate}
\item \emph{Type I Error}: Reject $H_{0}$ when it was true. 
\item \emph{Type II Error}: Not rejecting $H_{0}$ when $H_{1}$ is true. 
\item \emph{Power}: Probability of rejecting $H_{0}$ when $H_{1}$ is true. 
\end{enumerate}

\paragraph{Samples from Two Populations}

Use \emph{bias-corrected pooled sample variance}: {\small{}
\[
S_{n_{1}+n_{2}-2}^{2}=\frac{\sum_{i=1}^{n_{1}}\left(X_{i}-\overline{X}\right)^{2}+\sum_{i=1}^{n_{2}}\left(Y_{i}-\overline{Y}\right)^{2}}{n_{1}+n_{2}-2}=\frac{n_{1}-1}{n_{1}+n_{2}-2}S_{n_{1}}^{2}+\frac{n_{2}-1}{n_{1}+n_{2}-2}S_{n_{2}}^{2}
\]
}{\small \par}


\subsection{Goodness of Fit}


\paragraph{Chi-Square Statistic}

\textrm{\textcolor{red}{
\[
\chi^{2}=\sum_{i=1}^{k}\frac{\left(O_{i}-E_{i}\right)^{2}}{E_{i}}
\]
}}
\begin{itemize}
\item \textcolor{red}{Approximation is valid only if $\forall jE_{j}\geq5$. }
\item The \emph{rejection region} at the $100\alpha\%$ level is given by
\[
R=\left\{ x^{2}\mid x^{2}>\chi_{k-p-1,1-\alpha}^{2}\right\} 
\]
where $k$ is the number of terms summed and $p$ is the number of
parameters being estimated. 
\end{itemize}

\subsection{Independence Testing}
\begin{enumerate}
\item Write up a contingency table. 
\begin{table}[H]
\centering{}%
\begin{tabular}{c|ccc|c}
 & $y_{1}$  & $\dots$  & $y_{l}$  & \tabularnewline
\hline 
$x_{1}$  & $n_{11}$  &  & $n_{1l}$  & $n_{1\bullet}$\tabularnewline
$\vdots$  &  &  &  & \tabularnewline
$x_{k}$  & $n_{k1}$  &  & $n_{kl}$  & $n_{k\bullet}$\tabularnewline
\hline 
 & $n_{\bullet1}$  &  & $n_{\bullet l}$  & $n$\tabularnewline
\end{tabular}
\end{table}

\item The expected value in each cell is $\hat{n}_{ij}=\frac{n_{i\bullet}\times n_{\bullet j}}{n}$. 
\item Compute the $x^{2}$ statistic. 
\item Compare against $\chi^{2}$ dist. with $kl-\left(k-1\right)-\left(l-1\right)-1=\left(k-1\right)\left(l-1\right)$
degrees of freedom. \end{enumerate}

\end{document}
